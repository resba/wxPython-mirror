%%%%%%%%%%%%%%%%%%%%%%%%%%%%%%%%%%%%%%%%%%%%%%%%%%%%%%%%%%%%%%%%%%%%%%%%%%%%%%%
%% Name:        mediaevt.tex
%% Purpose:     wxMediaEvent docs
%% Author:      Ryan Norton <wxprojects@comcast.net>
%% Modified by:
%% Created:     11/7/2004
%% RCS-ID:      $Id: mediaevt.tex 44524 2007-02-19 18:30:26Z JS $
%% Copyright:   (c) Ryan Norton
%% License:     wxWindows license
%%%%%%%%%%%%%%%%%%%%%%%%%%%%%%%%%%%%%%%%%%%%%%%%%%%%%%%%%%%%%%%%%%%%%%%%%%%%%%%

\section{\class{wxMediaEvent}}\label{wxmediaevent}

Event \helpref{wxMediaCtrl}{wxmediactrl} uses.

\wxheading{Derived from}

\helpref{wxNotifyEvent}{wxnotifyevent}

\wxheading{Include files}

<wx/mediactrl.h>

\wxheading{Event table macros}

\twocolwidtha{7cm}
\begin{twocollist}\itemsep=0pt
\twocolitem{{\bf EVT\_MEDIA\_LOADED(id, func)}}{
Sent when a media has loaded enough data that it can start playing.}
\twocolitem{{\bf EVT\_MEDIA\_STOP(id, func)}}{
Send when a media has switched to the wxMEDIASTATE\_STOPPED state.
You may be able to Veto this event to prevent it from stopping,
causing it to continue playing - even if it has reached that end of the media
(note that this may not have the desired effect - if you want to loop the
media, for example, catch the EVT\_MEDIA\_FINISHED and play there instead). }
\twocolitem{{\bf EVT\_MEDIA\_FINISHED(id, func)}}{
Sent when a media has finished playing in a \helpref{wxMediaCtrl}{wxmediactrl}.
}
\twocolitem{{\bf EVT\_MEDIA\_STATECHANGED(id, func)}}{
Send when a media has switched its state (from any media state).
}
\twocolitem{{\bf EVT\_MEDIA\_PLAY(id, func)}}{
Send when a media has switched to the wxMEDIASTATE\_PLAYING state.
}
\twocolitem{{\bf EVT\_MEDIA\_PAUSE(id, func)}}{
Send when a media has switched to the wxMEDIASTATE\_PAUSED state.
}
\end{twocollist}

\latexignore{\rtfignore{\wxheading{Members}}}

