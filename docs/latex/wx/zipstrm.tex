%
% automatically generated by HelpGen $Revision: 43373 $ from
% wx/zipstrm.h at 16/Sep/04 12:19:29
%

\section{\class{wxZipClassFactory}}\label{wxzipclassfactory}

Class factory for the zip archive format. See the base class
for details.

\wxheading{Derived from}

\helpref{wxArchiveClassFactory}{wxarchiveclassfactory}

\wxheading{Include files}

<wx/zipstrm.h>

\wxheading{See also}

\helpref{Archive formats such as zip}{wxarc}\\
\helpref{Generic archive programming}{wxarcgeneric}\\
\helpref{wxZipEntry}{wxzipentry}\\
\helpref{wxZipInputStream}{wxzipinputstream}\\
\helpref{wxZipOutputStream}{wxzipoutputstream}


%
% automatically generated by HelpGen $Revision: 43373 $ from
% wx/zipstrm.h at 16/Sep/04 12:19:29
%

\section{\class{wxZipEntry}}\label{wxzipentry}

Holds the meta-data for an entry in a zip.

\wxheading{Derived from}

\helpref{wxArchiveEntry}{wxarchiveentry}

\wxheading{Include files}

<wx/zipstrm.h>

\wxheading{Data structures}

Constants for \helpref{Get/SetMethod}{wxzipentrymethod}:

\begin{verbatim}
// Compression Method, only 0 (store) and 8 (deflate) are supported here
//
enum wxZipMethod
{
    wxZIP_METHOD_STORE,
    wxZIP_METHOD_SHRINK,
    wxZIP_METHOD_REDUCE1,
    wxZIP_METHOD_REDUCE2,
    wxZIP_METHOD_REDUCE3,
    wxZIP_METHOD_REDUCE4,
    wxZIP_METHOD_IMPLODE,
    wxZIP_METHOD_TOKENIZE,
    wxZIP_METHOD_DEFLATE,
    wxZIP_METHOD_DEFLATE64,
    wxZIP_METHOD_BZIP2 = 12,
    wxZIP_METHOD_DEFAULT = 0xffff
};

\end{verbatim}

Constants for \helpref{Get/SetSystemMadeBy}{wxzipentrysystemmadeby}:

\begin{verbatim}
// Originating File-System.
// 
// These are Pkware's values. Note that Info-zip disagree on some of them,
// most notably NTFS.
//
enum wxZipSystem
{
    wxZIP_SYSTEM_MSDOS,
    wxZIP_SYSTEM_AMIGA,
    wxZIP_SYSTEM_OPENVMS,
    wxZIP_SYSTEM_UNIX,
    wxZIP_SYSTEM_VM_CMS,
    wxZIP_SYSTEM_ATARI_ST,
    wxZIP_SYSTEM_OS2_HPFS,
    wxZIP_SYSTEM_MACINTOSH,
    wxZIP_SYSTEM_Z_SYSTEM,
    wxZIP_SYSTEM_CPM,
    wxZIP_SYSTEM_WINDOWS_NTFS,
    wxZIP_SYSTEM_MVS,
    wxZIP_SYSTEM_VSE,
    wxZIP_SYSTEM_ACORN_RISC,
    wxZIP_SYSTEM_VFAT,
    wxZIP_SYSTEM_ALTERNATE_MVS,
    wxZIP_SYSTEM_BEOS,
    wxZIP_SYSTEM_TANDEM,
    wxZIP_SYSTEM_OS_400
};

\end{verbatim}

Constants for \helpref{Get/SetExternalAttributes}{wxzipentryexternalattributes}:

\begin{verbatim}
// Dos/Win file attributes
//
enum wxZipAttributes
{
    wxZIP_A_RDONLY = 0x01,
    wxZIP_A_HIDDEN = 0x02,
    wxZIP_A_SYSTEM = 0x04,
    wxZIP_A_SUBDIR = 0x10,
    wxZIP_A_ARCH   = 0x20,

    wxZIP_A_MASK   = 0x37
};

\end{verbatim}

Constants for \helpref{Get/SetFlags}{wxzipentrygetflags}:

\begin{verbatim}
// Values for the flags field in the zip headers
//
enum wxZipFlags
{
    wxZIP_ENCRYPTED         = 0x0001,
    wxZIP_DEFLATE_NORMAL    = 0x0000,   // normal compression
    wxZIP_DEFLATE_EXTRA     = 0x0002,   // extra compression
    wxZIP_DEFLATE_FAST      = 0x0004,   // fast compression
    wxZIP_DEFLATE_SUPERFAST = 0x0006,   // superfast compression
    wxZIP_DEFLATE_MASK      = 0x0006,
    wxZIP_SUMS_FOLLOW       = 0x0008,   // crc and sizes come after the data
    wxZIP_ENHANCED          = 0x0010,
    wxZIP_PATCH             = 0x0020,
    wxZIP_STRONG_ENC        = 0x0040,
    wxZIP_UNUSED            = 0x0F80,
    wxZIP_RESERVED          = 0xF000
};

\end{verbatim}

\wxheading{See also}

\helpref{Archive formats such as zip}{wxarc}\\
\helpref{wxZipInputStream}{wxzipinputstream}\\
\helpref{wxZipOutputStream}{wxzipoutputstream}\\
\helpref{wxZipNotifier}{wxzipnotifier}

\wxheading{Field availability}

When reading a zip from a stream that is seekable,
 \helpref{GetNextEntry()}{wxzipinputstreamgetnextentry} returns
a fully populated wxZipEntry object except for
 \helpref{wxZipEntry::GetLocalExtra()}{wxzipentrylocalextra}. GetLocalExtra()
becomes available when the entry is opened, either by calling
 \helpref{wxZipInputStream::OpenEntry}{wxzipinputstreamopenentry} or by
making an attempt to read the entry's data.

For zips on \helpref{non-seekable}{wxarcnoseek} streams, the following
fields are always available when GetNextEntry() returns:

\helpref{GetDateTime}{wxarchiveentrydatetime}\\
\helpref{GetInternalFormat}{wxarchiveentrygetinternalformat}\\
\helpref{GetInternalName}{wxzipentrygetinternalname}\\
\helpref{GetFlags}{wxzipentrygetflags}\\
\helpref{GetLocalExtra}{wxzipentrylocalextra}\\
\helpref{GetMethod}{wxzipentrymethod}\\
\helpref{GetName}{wxarchiveentryname}\\
\helpref{GetOffset}{wxarchiveentrygetoffset}\\
\helpref{IsDir}{wxarchiveentryisdir}

The following fields are also usually available when GetNextEntry()
returns, however, if the zip was also written to a non-seekable stream
the zipper is permitted to store them after the entry's data. In that
case they become available when the entry's data has been read to Eof(),
or \helpref{CloseEntry()}{wxarchiveinputstreamcloseentry} has been called.
{\tt (GetFlags() \& wxZIP\_SUMS\_FOLLOW) != 0} indicates that one or
more of these come after the data:

\helpref{GetCompressedSize}{wxzipentrygetcompressedsize}\\
\helpref{GetCrc}{wxzipentrygetcrc}\\
\helpref{GetSize}{wxarchiveentrysize}

The following are stored at the end of the zip, and become available
when the end of the zip has been reached, i.e. after GetNextEntry()
returns NULL and Eof() is true:

\helpref{GetComment}{wxzipentrycomment}\\
\helpref{GetExternalAttributes}{wxzipentryexternalattributes}\\
\helpref{GetExtra}{wxzipentryextra}\\
\helpref{GetMode}{wxzipentrymode}\\
\helpref{GetSystemMadeBy}{wxzipentrysystemmadeby}\\
\helpref{IsReadOnly}{wxarchiveentryisreadonly}\\
\helpref{IsMadeByUnix}{wxzipentryismadebyunix}\\
\helpref{IsText}{wxzipentryistext}


\latexignore{\rtfignore{\wxheading{Members}}}


\membersection{wxZipEntry::wxZipEntry}\label{wxzipentrywxzipentry}

\func{}{wxZipEntry}{\param{const wxString\& }{name = wxEmptyString}, \param{const wxDateTime\& }{dt = wxDateTime::Now()}, \param{off\_t }{size = wxInvalidOffset}}

Constructor.

\func{}{wxZipEntry}{\param{const wxZipEntry\& }{entry}}

Copy constructor.


\membersection{wxZipEntry::Clone}\label{wxzipentryclone}

\constfunc{wxZipEntry*}{Clone}{\void}

Make a copy of this entry.


\membersection{wxZipEntry::Get/SetComment}\label{wxzipentrycomment}

\constfunc{wxString}{GetComment}{\void}

\func{void}{SetComment}{\param{const wxString\& }{comment}}

A short comment for this entry.


\membersection{wxZipEntry::GetCompressedSize}\label{wxzipentrygetcompressedsize}

\constfunc{off\_t}{GetCompressedSize}{\void}

The compressed size of this entry in bytes.


\membersection{wxZipEntry::GetCrc}\label{wxzipentrygetcrc}

\constfunc{wxUint32}{GetCrc}{\void}

CRC32 for this entry's data.


\membersection{wxZipEntry::Get/SetExternalAttributes}\label{wxzipentryexternalattributes}

\constfunc{wxUint32}{GetExternalAttributes}{\void}

\func{void}{SetExternalAttributes}{\param{wxUint32 }{attr}}

The low 8 bits are always the DOS/Windows file attributes for this entry.
The values of these attributes are given in the
enumeration {\tt wxZipAttributes}.

The remaining bits can store platform specific permission bits or
attributes, and their meaning depends on the value
of \helpref{SetSystemMadeBy()}{wxzipentrysystemmadeby}.
If \helpref{IsMadeByUnix()}{wxzipentryismadebyunix} is true then the
high 16 bits are unix mode bits.

The following other accessors access these bits:

\helpref{IsReadOnly/SetIsReadOnly}{wxarchiveentryisreadonly}\\
\helpref{IsDir/SetIsDir}{wxarchiveentryisdir}\\
\helpref{Get/SetMode}{wxzipentrymode}


\membersection{wxZipEntry::Get/SetExtra}\label{wxzipentryextra}

\constfunc{const char*}{GetExtra}{\void}

\constfunc{size\_t}{GetExtraLen}{\void}

\func{void}{SetExtra}{\param{const char* }{extra}, \param{size\_t }{len}}

The extra field from the entry's central directory record.

The extra field is used to store platform or application specific
data. See Pkware's document 'appnote.txt' for information on its format.


\membersection{wxZipEntry::GetFlags}\label{wxzipentrygetflags}

\constfunc{int}{GetFlags}{\void}

Returns a combination of the bits flags in the enumeration {\tt wxZipFlags}.


\membersection{wxZipEntry::GetInternalName}\label{wxzipentrygetinternalname}

\constfunc{wxString}{GetInternalName}{\void}

Returns the entry's filename in the internal format used within the
archive. The name can include directory components, i.e. it can be a
full path.

The names of directory entries are returned without any trailing path
separator. This gives a canonical name that can be used in comparisons.

\func{wxString}{GetInternalName}{\param{const wxString\& }{name}, \param{wxPathFormat }{format = wxPATH\_NATIVE}, \param{bool* }{pIsDir = NULL}}

A static member that translates a filename into the internal format used
within the archive. If the third parameter is provided, the bool pointed
to is set to indicate whether the name looks like a directory name
(i.e. has a trailing path separator).

\wxheading{See also}

\helpref{Looking up an archive entry by name}{wxarcbyname}


\membersection{wxZipEntry::Get/SetLocalExtra}\label{wxzipentrylocalextra}

\constfunc{const char*}{GetLocalExtra}{\void}

\constfunc{size\_t}{GetLocalExtraLen}{\void}

\func{void}{SetLocalExtra}{\param{const char* }{extra}, \param{size\_t }{len}}

The extra field from the entry's local record.

The extra field is used to store platform or application specific
data. See Pkware's document 'appnote.txt' for information on its format.


\membersection{wxZipEntry::Get/SetMethod}\label{wxzipentrymethod}

\constfunc{int}{GetMethod}{\void}

\func{void}{SetMethod}{\param{int }{method}}

The compression method. The enumeration {\tt wxZipMethod} lists the
possible values.

The default constructor sets this to wxZIP\_METHOD\_DEFAULT,
which allows \helpref{wxZipOutputStream}{wxzipoutputstream} to
choose the method when writing the entry.


\membersection{wxZipEntry::Get/SetMode}\label{wxzipentrymode}

\constfunc{int}{GetMode}{\void}

If \helpref{IsMadeByUnix()}{wxzipentryismadebyunix} is true then
returns the unix permission bits stored in
 \helpref{GetExternalAttributes()}{wxzipentryexternalattributes}.
Otherwise synthesises them from the DOS attributes.

\func{void}{SetMode}{\param{int }{mode}}

Sets the DOS attributes
in \helpref{GetExternalAttributes()}{wxzipentryexternalattributes}
to be consistent with the {\tt mode} given.

If \helpref{IsMadeByUnix()}{wxzipentryismadebyunix} is true then also
stores {\tt mode} in GetExternalAttributes().

Note that the default constructor
sets \helpref{GetSystemMadeBy()}{wxzipentrysystemmadeby} to 
wxZIP\_SYSTEM\_MSDOS by default. So to be able to store unix
permissions when creating zips, call SetSystemMadeBy(wxZIP\_SYSTEM\_UNIX).


\membersection{wxZipEntry::SetNotifier}\label{wxzipentrynotifier}

\func{void}{SetNotifier}{\param{wxZipNotifier\& }{notifier}}

\func{void}{UnsetNotifier}{\void}

Sets the \helpref{notifier}{wxzipnotifier} for this entry.
Whenever the \helpref{wxZipInputStream}{wxzipinputstream} updates
this entry, it will then invoke the associated
notifier's \helpref{OnEntryUpdated}{wxzipnotifieronentryupdated}
method.

Setting a notifier is not usually necessary. It is used to handle
certain cases when modifying an zip in a pipeline (i.e. between
non-seekable streams).

\wxheading{See also}

\helpref{Archives on non-seekable streams}{wxarcnoseek}\\
\helpref{wxZipNotifier}{wxzipnotifier}


\membersection{wxZipEntry::Get/SetSystemMadeBy}\label{wxzipentrysystemmadeby}

\constfunc{int}{GetSystemMadeBy}{\void}

\func{void}{SetSystemMadeBy}{\param{int }{system}}

The originating file-system. The default constructor sets this to
wxZIP\_SYSTEM\_MSDOS. Set it to wxZIP\_SYSTEM\_UNIX in order to be
able to store unix permissions using \helpref{SetMode()}{wxzipentrymode}.


\membersection{wxZipEntry::IsMadeByUnix}\label{wxzipentryismadebyunix}

\constfunc{bool}{IsMadeByUnix}{\void}

Returns true if \helpref{GetSystemMadeBy()}{wxzipentrysystemmadeby}
is a flavour of unix.


\membersection{wxZipEntry::IsText/SetIsText}\label{wxzipentryistext}

\constfunc{bool}{IsText}{\void}

\func{void}{SetIsText}{\param{bool }{isText = true}}

Indicates that this entry's data is text in an 8-bit encoding.


\membersection{wxZipEntry::operator=}\label{wxzipentryoperatorassign}

\func{wxZipEntry\& operator}{operator=}{\param{const wxZipEntry\& }{entry}}

Assignment operator.


%
% automatically generated by HelpGen $Revision: 43373 $ from
% wx/zipstrm.h at 16/Sep/04 12:19:29
%

\section{\class{wxZipInputStream}}\label{wxzipinputstream}

Input stream for reading zip files.

\helpref{GetNextEntry()}{wxzipinputstreamgetnextentry} returns an
 \helpref{wxZipEntry}{wxzipentry} object containing the meta-data
for the next entry in the zip (and gives away ownership). Reading from
the wxZipInputStream then returns the entry's data. Eof() becomes true
after an attempt has been made to read past the end of the entry's data.
When there are no more entries, GetNextEntry() returns NULL and sets Eof().

Note that in general zip entries are not seekable, and
wxZipInputStream::SeekI() always returns wxInvalidOffset.

\wxheading{Derived from}

\helpref{wxArchiveInputStream}{wxarchiveinputstream}

\wxheading{Include files}

<wx/zipstrm.h>

\wxheading{Data structures}
\begin{verbatim}
typedef wxZipEntry entry_type
\end{verbatim}

\wxheading{See also}

\helpref{Archive formats such as zip}{wxarc}\\
\helpref{wxZipEntry}{wxzipentry}\\
\helpref{wxZipOutputStream}{wxzipoutputstream}

\latexignore{\rtfignore{\wxheading{Members}}}


\membersection{wxZipInputStream::wxZipInputStream}\label{wxzipinputstreamwxzipinputstream}

\func{}{wxZipInputStream}{\param{wxInputStream\& }{stream}, \param{wxMBConv\& }{conv = wxConvLocal}}

\func{}{wxZipInputStream}{\param{wxInputStream*}{stream}, \param{wxMBConv\& }{conv = wxConvLocal}}

Constructor. In a Unicode build the second parameter {\tt conv} is
used to translate the filename and comment fields into Unicode. It has
no effect on the stream's data.

If the parent stream is passed as a pointer then the new filter stream
takes ownership of it. If it is passed by reference then it does not.

\func{}{wxZipInputStream}{\param{const wxString\& }{archive}, \param{const wxString\& }{file}}

Compatibility constructor (requires WXWIN\_COMPATIBILITY\_2\_6).

When this constructor is used, an emulation of seeking is
switched on for compatibility with previous versions. Note however,
that it is deprecated.


\membersection{wxZipInputStream::CloseEntry}\label{wxzipinputstreamcloseentry}

\func{bool}{CloseEntry}{\void}

Closes the current entry. On a non-seekable stream reads to the end of
the current entry first.


\membersection{wxZipInputStream::GetComment}\label{wxzipinputstreamgetcomment}

\func{wxString}{GetComment}{\void}

Returns the zip comment.

This is stored at the end of the zip, therefore when reading a zip
from a non-seekable stream, it returns the empty string until the
end of the zip has been reached, i.e. when GetNextEntry() returns
NULL.


\membersection{wxZipInputStream::GetNextEntry}\label{wxzipinputstreamgetnextentry}

\func{wxZipEntry*}{GetNextEntry}{\void}

Closes the current entry if one is open, then reads the meta-data for
the next entry and returns it in a \helpref{wxZipEntry}{wxzipentry}
object, giving away ownership. The stream is then open and can be read.


\membersection{wxZipInputStream::GetTotalEntries}\label{wxzipinputstreamgettotalentries}

\func{int}{GetTotalEntries}{\void}

For a zip on a seekable stream returns the total number of entries in
the zip. For zips on non-seekable streams returns the number of entries
returned so far by \helpref{GetNextEntry()}{wxzipinputstreamgetnextentry}.


\membersection{wxZipInputStream::OpenEntry}\label{wxzipinputstreamopenentry}

\func{bool}{OpenEntry}{\param{wxZipEntry\& }{entry}}

Closes the current entry if one is open, then opens the entry specified
by the {\it entry} object.

{\it entry} should be from the same zip file, and the zip should
be on a seekable stream.

\wxheading{See also}

\helpref{Looking up an archive entry by name}{wxarcbyname}


%
% automatically generated by HelpGen $Revision: 43373 $ from
% wx/zipstrm.h at 16/Sep/04 12:19:29
%

\section{\class{wxZipNotifier}}\label{wxzipnotifier}

If you need to know when a \helpref{wxZipInputStream}{wxzipinputstream}
updates a \helpref{wxZipEntry}{wxzipentry},
you can create a notifier by deriving from this abstract base class,
overriding \helpref{OnEntryUpdated()}{wxzipnotifieronentryupdated}.
An instance of your notifier class can then be assigned to wxZipEntry
objects, using \helpref{wxZipEntry::SetNotifier()}{wxzipentrynotifier}.

Setting a notifier is not usually necessary. It is used to handle
certain cases when modifying an zip in a pipeline (i.e. between
non-seekable streams).
See '\helpref{Archives on non-seekable streams}{wxarcnoseek}'.

\wxheading{Derived from}

No base class

\wxheading{Include files}

<wx/zipstrm.h>

\wxheading{See also}

\helpref{Archives on non-seekable streams}{wxarcnoseek}\\
\helpref{wxZipEntry}{wxzipentry}\\
\helpref{wxZipInputStream}{wxzipinputstream}\\
\helpref{wxZipOutputStream}{wxzipoutputstream}

\latexignore{\rtfignore{\wxheading{Members}}}


\membersection{wxZipNotifier::OnEntryUpdated}\label{wxzipnotifieronentryupdated}

\func{void}{OnEntryUpdated}{\param{wxZipEntry\& }{entry}}

Override this to receive notifications when
an \helpref{wxZipEntry}{wxzipentry} object changes.


%
% automatically generated by HelpGen $Revision: 43373 $ from
% wx/zipstrm.h at 16/Sep/04 12:19:29
%

\section{\class{wxZipOutputStream}}\label{wxzipoutputstream}

Output stream for writing zip files.

\helpref{PutNextEntry()}{wxzipoutputstreamputnextentry} is used to create
a new entry in the output zip, then the entry's data is written to the
wxZipOutputStream.  Another call to PutNextEntry() closes the current
entry and begins the next.

\wxheading{Derived from}

\helpref{wxArchiveOutputStream}{wxarchiveoutputstream}

\wxheading{Include files}

<wx/zipstrm.h>

\wxheading{See also}

\helpref{Archive formats such as zip}{wxarc}\\
\helpref{wxZipEntry}{wxzipentry}\\
\helpref{wxZipInputStream}{wxzipinputstream}

\latexignore{\rtfignore{\wxheading{Members}}}


\membersection{wxZipOutputStream::wxZipOutputStream}\label{wxzipoutputstreamwxzipoutputstream}

\func{}{wxZipOutputStream}{\param{wxOutputStream\& }{stream}, \param{int }{level = -1}, \param{wxMBConv\& }{conv = wxConvLocal}}

\func{}{wxZipOutputStream}{\param{wxOutputStream*}{stream}, \param{int }{level = -1}, \param{wxMBConv\& }{conv = wxConvLocal}}

Constructor. {\tt level} is the compression level to use.
It can be a value between 0 and 9 or -1 to use the default value
which currently is equivalent to 6.

If the parent stream is passed as a pointer then the new filter stream
takes ownership of it. If it is passed by reference then it does not.

In a Unicode build the third parameter {\tt conv} is used to translate
the filename and comment fields to an 8-bit encoding. It has no effect on the
stream's data.


\membersection{wxZipOutputStream::\destruct{wxZipOutputStream}}\label{wxzipoutputstreamdtor}

\func{}{\destruct{wxZipOutputStream}}{\void}

The destructor calls \helpref{Close()}{wxzipoutputstreamclose} to finish
writing the zip if it has not been called already.


\membersection{wxZipOutputStream::Close}\label{wxzipoutputstreamclose}

\func{bool}{Close}{\void}

Finishes writing the zip, returning true if successful.
Called by the destructor if not called explicitly.


\membersection{wxZipOutputStream::CloseEntry}\label{wxzipoutputstreamcloseentry}

\func{bool}{CloseEntry}{\void}

Close the current entry. It is called implicitly whenever another new
entry is created with \helpref{CopyEntry()}{wxzipoutputstreamcopyentry}
or \helpref{PutNextEntry()}{wxzipoutputstreamputnextentry}, or
when the zip is closed.


\membersection{wxZipOutputStream::CopyArchiveMetaData}\label{wxzipoutputstreamcopyarchivemetadata}

\func{bool}{CopyArchiveMetaData}{\param{wxZipInputStream\& }{inputStream}}

Transfers the zip comment from the \helpref{wxZipInputStream}{wxzipinputstream}
to this output stream.


\membersection{wxZipOutputStream::CopyEntry}\label{wxzipoutputstreamcopyentry}

\func{bool}{CopyEntry}{\param{wxZipEntry* }{entry}, \param{wxZipInputStream\& }{inputStream}}

Takes ownership of {\tt entry} and uses it to create a new entry
in the zip. {\tt entry} is then opened in {\tt inputStream} and its contents
copied to this stream.

CopyEntry() is much more efficient than transferring the data using
Read() and Write() since it will copy them without decompressing and
recompressing them.

For zips on seekable streams, {\tt entry} must be from the same zip file
as {\tt stream}. For non-seekable streams, {\tt entry} must also be the
last thing read from {\tt inputStream}.


\membersection{wxZipOutputStream::Get/SetLevel}\label{wxzipoutputstreamlevel}

\constfunc{int}{GetLevel}{\void}

\func{void}{SetLevel}{\param{int }{level}}

Set the compression level that will be used the next time an entry is
created. It can be a value between 0 and 9 or -1 to use the default value
which currently is equivalent to 6.


\membersection{wxZipOutputStream::PutNextDirEntry}\label{wxzipoutputstreamputnextdirentry}

\func{bool}{PutNextDirEntry}{\param{const wxString\& }{name}, \param{const wxDateTime\& }{dt = wxDateTime::Now()}}

Create a new directory entry
(see \helpref{wxArchiveEntry::IsDir()}{wxarchiveentryisdir})
with the given name and timestamp.

\helpref{PutNextEntry()}{wxzipoutputstreamputnextentry} can
also be used to create directory entries, by supplying a name with
a trailing path separator.


\membersection{wxZipOutputStream::PutNextEntry}\label{wxzipoutputstreamputnextentry}

\func{bool}{PutNextEntry}{\param{wxZipEntry* }{entry}}

Takes ownership of {\tt entry} and uses it to create a new entry
in the zip. 

\func{bool}{PutNextEntry}{\param{const wxString\& }{name}, \param{const wxDateTime\& }{dt = wxDateTime::Now()}, \param{off\_t }{size = wxInvalidOffset}}

Create a new entry with the given name, timestamp and size.


\membersection{wxZipOutputStream::SetComment}\label{wxzipoutputstreamsetcomment}

\func{void}{SetComment}{\param{const wxString\& }{comment}}

Sets a comment for the zip as a whole. It is written at the end of the
zip.

